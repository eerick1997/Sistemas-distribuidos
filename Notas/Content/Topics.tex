\chapter{Introducción}
Asignatura: Desarrollo de sisteas distribuidos
Prof. Ukranio

\section{Evaluación}

\begin{itemize}
    \item {Portafolio de evidencias: 35\%}
    \item {Examen 15\%}
    \item {Proyecto 50\%}
    \item {Libro +10\%}
\end{itemize}

\section{Material}
\begin{itemize}
    \item {Laptop con alguna distribución de linux}
    \item {Sitio ukranio.gnomio.com}
\end{itemize}

\section{Comandos más importantes Unix}

\begin{longtable}[c]{@{}
    >{\columncolor[HTML]{00009B}}l l@{}}
    \toprule
    {\color[HTML]{FFFFFF} Comando} & \cellcolor[HTML]{00009B}{\color[HTML]{FFFFFF} Acción}                                      \\* \midrule
    \endfirsthead
    %
    \endhead
    %
    {\color[HTML]{FFFFFF} cp}      & \cellcolor[HTML]{FFFFFF}{\color[HTML]{333333} copiar archivo}                              \\* \midrule
    {\color[HTML]{FFFFFF} mv}      & \cellcolor[HTML]{FFFFFF}{\color[HTML]{333333} mover o renombrar archivos o directorios}    \\* \midrule
    {\color[HTML]{FFFFFF} gzip}    & \cellcolor[HTML]{FFFFFF}{\color[HTML]{333333} comprime un archivo}                         \\* \midrule
    {\color[HTML]{FFFFFF} ftp}     & \cellcolor[HTML]{FFFFFF}{\color[HTML]{333333} transfiere el archivo de un programa}        \\* \midrule
    {\color[HTML]{FFFFFF} lpr}     & imprime la salida de un archivo                                                            \\* \midrule
    {\color[HTML]{FFFFFF} mkdir}   & crea un directorio                                                                         \\* \midrule
    {\color[HTML]{FFFFFF} rm}      & remueve archivos o directorios                                                             \\* \midrule
    {\color[HTML]{FFFFFF} rmdir}   & remueve un directorio                                                                      \\* \midrule
    {\color[HTML]{FFFFFF} cd}      & cambia de directorio                                                                       \\* \midrule
    {\color[HTML]{FFFFFF} pwd}     & muestra el nombre del directorio actual                                                    \\* \midrule
    {\color[HTML]{FFFFFF} ls}      & lista los nombres de los archivos en un directorio                                         \\* \midrule
    {\color[HTML]{FFFFFF} cat}     & concatena y muestra archivos                                                               \\* \midrule
    {\color[HTML]{FFFFFF} find}    & encuentra archivos de un nombre específico o tipo                                          \\* \midrule
    {\color[HTML]{FFFFFF} grep}    & busca archivos por una cadena específica o expresión                                       \\* \midrule
    {\color[HTML]{FFFFFF} top}     & muestra el top 10 de los procesos del sistema y periódicamente actualiza esta información. \\* \midrule
    {\color[HTML]{FFFFFF} chmod}   & cambia los permisos de un archivo o directorio                                             \\* \midrule
    {\color[HTML]{FFFFFF} date}    & muestra la fecha actual                                                                    \\* \midrule
    {\color[HTML]{FFFFFF} man}     & muestra información de los manuales de referencia                                          \\* \midrule
    {\color[HTML]{FFFFFF} du}      & muestra el tamaño/uso de la carpeta en la que se está.                                     \\* \midrule
    {\color[HTML]{FFFFFF} df}      & reporta el archivo del sistema del uso del espacio en disco.                               \\* \bottomrule
\end{longtable}

\section{¿Qué es un sistema distribuido?}
Un componente fundamental en los sitemas distribuidos son las redes de cómputo.\\
Algunas aplicaciones que son aplicaciones distribuidos
\begin{itemize}
    \item {Ebay}
    \item {Amazon}
    \item {Paypal}
    \item {Facebook}
    \item {Whatsapp}
    \item {Twitter}
    \item {Instagram}
    \item {Youtube}
    \item {Wikipedia}
    \item {Google search (El segundo más grande del mundo)}
    \item {Internet (La aplicación más grande del mundo)}
    \item {Netflix}
    \item {Spotify}
    \item {Moodle}
    \item {Operaciones remotas}
    \item {Google maps}
    \item {Mapquest}
    \item {Uber}
\end{itemize}
En resumen aplicaciones de compra en linea, redes sociales, la misma internet, videojuegos on-line, cloud computing, big data, clusters de computadoras.

\subsection{Conceptos fundamentales de sistemas distribuidos}
\subsubsection{Heterogeneidad}
Implementa una diversidad de software y hardware, por ejemplos diferentes sistemas operativos, dispositivos, etc. Los sistemas distribuidos actuales en su mayoría son heterogeneos, esto les permite tener más alcance.

\subsubsection{Escalabilidad}
Un sistema puede ir creciendo o incrementandose sin hacer que crezca el sistema por completo, además el sistema sigue funcionando aún que crezca el servicio, nunca se detiene.

\subsubsection{Transparencia}
Tenemos varios tipos de transparencia
\begin{itemize}
    \item {Transparencia de acceso: por ejemplo si usamos dropbox sabemos cuando hay archivos en la nube o estan en nuestro dispositivo, es decir podemos diferenciar entre ambos.}
    \item {Transparencia de réplica: Algunos sistemas distribuidos replican información, por ejemplo Netflix hace esto en varios servidores con los vídeos que son más vistos}
    \item {Transparencia de localización: desconozco donde se encuentra algún recurso, por ejemplo si entramos a \textbf{www.ipn.mx} no está en una sola computadora sino en varias, se usa un servidor DNS (mapea un nombre a una IP).}
\end{itemize}

\subsubsection{Confiabilidad}
Hay seguridad, confiabilidad y estabilidad, esto se ha trasladado a muchas otras áreas.

\subsubsection{Middleware}
Nace al tener la idea de usar un software para crear aplicaciones distribuidas. Es un intermediario. Esto hace que la interfaz de programación sea la misma. Dos ejemplos de esto es Java RMI y CORBA - RMI.

\section{La arquitectura google}
Google es inventado por E.U.A. y se ubica en California.\\\\

\textbf{Misión}
Organizar la información mundial y hacerla accesible y usable universalmente.

\textbf{Origen}
Los algoritmos se originan en la universidad de stanford para dar resultados en menos de medio segundo.

\textbf{Infraestructura de harware}
\begin{itemize}
    \item {Computadoras prsonales}
    \item {2TB en disco duro}
    \item {16GB de RAM}
    \item {Una versión del Kernel Linux}
\end{itemize}

Se estima que google tiene 200 clusters con una cantidad de 1 exabyte de RAM es decir $10^{18}$ bytes

\section{Fallas en el sistema Google}
La falla más frecuente se debe a software al rededor de 20 PCs se resetean al día. Las fallas en hardware representan $\frac{1}{10}$ fallas.

\section{Puntos fundamentales de diseño}
\begin{itemize}
    \item {Simplicidad: el software debería hacer una cosa y hacerla bien, evitando los diseños multipropósito siempre que sea posible}
    \item {Desempeño: hacer énfasis en el desempeño del software compendiado en la frase ''cada milisegundo cuenta''}
    \item {Sin fallas: implementar pruebas rigurosas de software. La frase correspondiente: ''si no se ha descompuesto, no estás intentando con suficiente desempeño''}
\end{itemize}

\subsection{¿Porqué se calienta una computadora?}
Por la fricción entre electrónes, cuando estamos en un pulso en alto se desplaza un electrón a una dirección y en un pulso en bajo a la dirección contraria.

\textbf{ifconfig } Linux

Transferencia de archivos servidor FTP.
\begin{lstlisting}
    FTP la-ip (10.100.75.12)
    login: alumno
    password: escom2019
\end{lstlisting}

